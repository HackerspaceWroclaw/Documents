\documentclass{article}
\usepackage[OT4]{fontenc}
\usepackage{polski}
\usepackage[utf8]{inputenc}

\renewcommand{\thesection}{§ \Roman{section}.}
\renewcommand{\labelenumi}{\arabic{enumi}.}
\renewcommand{\labelenumii}{\arabic{enumi}.\arabic{enumii}.}
\renewcommand{\labelenumiii}{\arabic{enumi}.\arabic{enumii}.\arabic{enumiii}}

\begin{document}
\section{POSTANOWIENIA OGÓLNE}
  \begin{enumerate}
    \item Stowarzyszenie nosi nazwę: Stowarzyszenie Hackerspace Wrocław. W dalszej części statutu jest nazywane Stowarzyszeniem.
    \item Stowarzyszenie jest zawiązane na czas nieograniczony. Posiada osobowość prawną od dnia wpisu do rejestru KRS. Działa na podstawie przepisów ustawy z dnia 7 kwietnia 1989 r. Prawo o Stowarzyszeniach oraz niniejszego statutu.
    \item Terenem działania Stowarzyszenia jest Rzeczpospolita Polska, a siedzibą miasto Wrocław.
    \item Dla realizacji celów statutowych stowarzyszenie może działać na terenie innych państw z poszanowaniem tamtejszego prawa.
    \item Stowarzyszenie może być członkiem innych krajowych i międzynarodowych organizacji o podobnych celach, na warunkach określonych w~ich statucie, o~ile nie narusza to zobowiązań z~umów międzynarodowych, których stroną jest RP.
    \item Działalność Stowarzyszenia oparta jest przede wszystkim na pracy społecznej członków. Do prowadzenia swych spraw stowarzyszenie może zatrudniać pracowników.
    \item Członkami Stowarzyszenia mogą być cudzoziemcy, włącznie z osobami nie mającymi miejsca zamieszkania na terytorium Rzeczypospolitej Polskiej.
  \end{enumerate}
\section{CELE I ŚRODKI DZIAŁANIA}
  \begin{enumerate}
    \item Celem Stowarzyszenia jest upowszechnianie wiedzy, wspieranie innowacyjności oraz rozwój interdyscyplinarnych badań, prac naukowo-badawczych, działań kulturalnych oraz artystycznych.
    \item Stowarzyszenie swe cele realizuje poprzez:
      \begin{enumerate}
        \item tworzenie i utrzymywanie infrastruktury stymulującej rozwój projektów i użyczanie potrzebnych narzędzi,
        \item realizację i wspieranie projektów naukowych,
        \item działalność edukacyjną, w szczególności prowadzenie spotkań, konferencji, seminariów, wykładów i szkoleń,
        \item organizowanie konkursów i imprez promocyjnych oraz działalność kulturalną, w szczególności organizowanie wystaw, prezentacji, projekcji, happeningów oraz udział w wydarzeniach promujących naukę, edukację, sztukę,
        \item działalność publicystyczną i wydawniczą,
        \item integrację środowiska akademickiego, naukowego, przemysłowego i artystycznego,
        \item współpracę z krajowymi i zagranicznymi organizacjami o podobnych celach.
      \end{enumerate}
  \end{enumerate}
\section{CZŁONKOWIE}
  \begin{enumerate}
    \item Członkami Stowarzyszenia mogą być tylko osoby fizyczne.
    \item Stowarzyszenie posiada członków:
      \begin{enumerate}
        \item zwyczajnych,
        \item honorowych.
      \end{enumerate}
    \item Członkiem Zwyczajnym Stowarzyszenia może być osoba fizyczna, która spełnia stosowne postanowienia ustawy Prawo o Stowarzyszeniach oraz wszystkie poniższe warunki:
      \begin{enumerate}
        \item złoży deklarację członkowską na piśmie,
        \item przedstawi pozytywną opinię co najmniej jednego członka zwyczajnego Stowarzyszenia
      \end{enumerate}
    \item Założyciele stają się członkami zwyczajnymi w~momencie zawiązania się Stowarzyszenia.
    \item Osoba fizyczna staje się członkiem zwyczajnym stowarzyszenia po zaakceptowaniu pisemnej deklaracji członkowskiej uchwałą Zarządu Stowarzyszenia.
    \item Członkiem Honorowym Stowarzyszenia może być osoba fizyczna, która spełnia stosowne postanowienia ustawy Prawo o Stowarzyszeniach oraz wszystkie poniższe warunki:
      \begin{enumerate}
        \item jest pełnoletnia,
        \item wyrazi pisemną zgodę na nadanie jej w/w tytułu,
        \item wniosła wybitny wkład w działalność i rozwój Stowarzyszenia lub posiada znaczący wkład w dziedzinach zgodnych z celami Stowarzyszenia,
        \item zostanie zgłoszona przez co najmniej 10 członków zwyczajnych Stowarzyszenia lub Zarząd.
      \end{enumerate}
    \item Osoba fizyczna staje się członkiem honorowym stowarzyszenia po zaakceptowaniu wniosku na Walnym Zebraniu Członków.
    \item Członkowie zwyczajni Stowarzyszenia mają prawo:
      \begin{enumerate}
        \item korzystania z dorobku, majątku i wszelkich form działalności Stowarzyszenia,
        \item udziału w zebraniach, wykładach oraz imprezach organizowanych przez Stowarzyszenie,
        \item zgłaszania wniosków dotyczących działalności Stowarzyszenia,
        \item biernego i czynnego uczestniczenia w wyborach do władz Stowarzyszenia.
      \end{enumerate}
    \item Członkowie zwyczajni Stowarzyszenia mają obowiązek:
      \begin{enumerate}
        \item aktywnego udziału w działalności Stowarzyszenia i realizacji jego celów,
        \item przestrzegania statutu i uchwał władz Stowarzyszenia,
        \item regularnego opłacania składek.
      \end{enumerate}
    \item Członkowie Honorowi Stowarzyszenia mają prawo:
      \begin{enumerate}
        \item korzystania z dorobku, majątku i wszelkich form działalności Stowarzyszenia,
        \item udziału w zebraniach, wykładach oraz imprezach organizowanych przez Stowarzyszenie,
        \item zgłaszania wniosków dotyczących działalności Stowarzyszenia,
        \item brania udziału z głosem doradczym w statutowych władzach Stowarzyszenia.
      \end{enumerate}
    \item Członkowie Honorowi Stowarzyszenia mają obowiązek:
      \begin{enumerate}
        \item przestrzegania statutu oraz uchwał władz Stowarzyszenia.
      \end{enumerate}
    \item Członkowie Honorowi Stowarzyszenia są zwolnieni ze składek członkowskich.
    \item Członkowie Honorowi Stowarzyszenia nie posiadają biernego i czynnego prawa wyborczego we władzach Stowarzyszania.
    \item Różnych rodzajów Członkostw nie można łączyć.
    \item Utrata członkostwa następuje na skutek:
      \begin{enumerate}
        \item pisemnej rezygnacji złożonej na ręce Zarządu,
        \item wykluczenia z grona Członków przez Zarząd z powodu:
          \begin{enumerate}
            \item łamania statutu lub nieprzestrzegania uchwał władz Stowarzyszenia,
            \item unikania lub notorycznego braku udziału w pracach Stowarzyszenia,
            \item łamania zasad współpracy ze Stowarzyszeniem,
            \item braku wpłat składek członkowskich za okres trzech miesięcy,
            \item łamania zasad współżycia społecznego,
            \item działania na szkodę Stowarzyszenia,
            \item śmierci członka.
          \end{enumerate}
        \item Utrata członkostwa następuje na podstawie uchwały Zarządu.
        \item Od uchwały Zarządu w sprawie pozbawienia członkostwa zainteresowanemu przysługuje odwołanie do Walnego Zebrania Członków.
          \begin{enumerate}
            \item Odwołanie powinno zostać przekazane Zarządowi w formie pisemnej w terminie 14 dni od chwili poinformowania zainteresowanego o treści uchwały Zarządu.
            \item Decyzja Walnego Zebrania Członków jest ostateczna i wchodzi w życie w trybie natychmiastowym.
          \end{enumerate}
      \end{enumerate}
  \end{enumerate}
\section{WŁADZE}
  \begin{enumerate}
    \item Władzami Stowarzyszenia są:
      \begin{enumerate}
        \item Walne Zebranie Członków,
        \item Zarząd,
        \item Komisja Rewizyjna.
      \end{enumerate}
    \item Wybieralnymi władzami Stowarzyszenia są:
      \begin{enumerate}
        \item Zarząd,
        \item Komisja Rewizyjna.
      \end{enumerate}
    \item Kadencja wszystkich władz wybieralnych Stowarzyszenia trwa rok i~jest wspólna.
    \item Uchwały wszystkich władz Stowarzyszenia zapadają zwykłą większością głosów przy obecności co najmniej połowy członków uprawnionych do głosowania, stanowiących kworum, chyba że dalsze postanowienia statutu stanowią inaczej.
    \item Uchwały o wyborze i odwoływaniu władz stowarzyszenia, zmianach statutu oraz rozwiązania stowarzyszenia zapadają większością 2/3 uprawnionych do głosowania, przy obecności co najmniej 1/2 członków zwyczajnych Stowarzyszania.
    \item Uchwały wszystkich władz Stowarzyszenia zapadają w trybie głosowania jawnego. Na wniosek co najmniej jednego uczestnika zebrania, będącego członkiem zwyczajnym, w trybie głosowania tajnego.
    \item Walne Zebranie Członków jest najwyższą władzą Stowarzyszenia. Jego zadaniami są:
      \begin{enumerate}
        \item określenie głównych kierunków działania i rozwoju Stowarzyszenia,
        \item uchwalanie zmian statutu,
        \item uchwalanie budżetu,
        \item wybór i odwoływanie wybieralnych władz Stowarzyszenia, a także ich pojedynczych członków,
        \item udzielanie Zarządowi absolutorium na wniosek Komisji Rewizyjnej,
        \item ustalanie wysokości i sposobu uiszczania składek członkowskich,
        \item rozpatrywanie i zatwierdzanie sprawozdań władz Stowarzyszenia:
          \begin{enumerate}
            \item W przypadku odrzucenia sprawozdania, Walne Zebranie Członków przygotowuje listę poprawek, które Zarząd ma obowiązek wprowadzić,
            \item Poprawione sprawozdanie Zarząd przekazuje w ciągu 7 dni Komisji Rewizyjnej, która zatwierdza jego zgodność z wytycznymi Walnego Zebrania Członków.
          \end{enumerate}
        \item rozpatrywanie wniosków i postulatów zgłoszonych przez członków Stowarzyszenia lub jego władze,
        \item rozpatrywanie odwołań od uchwał Zarządu,
        \item podejmowanie uchwały o rozwiązaniu Stowarzyszenia i przeznaczeniu jego majątku,
        \item przyjmowanie i skreślanie członków honorowych,
        \item podejmowanie uchwał w każdej sprawie wniesionej pod obrady, we wszystkich sprawach nie zastrzeżonych do kompetencji innych władz Stowarzyszenia.
      \end{enumerate}
        \item W Walnym Zebraniu Członków biorą udział:
          \begin{enumerate}
            \item z głosem stanowiącym – członkowie zwyczajni,
            \item z głosem doradczym – członkowie honorowi oraz zaproszeni goście.
          \end{enumerate}
        \item Walne Zebranie Członków może być zwoływane w trybie zwyczajnym i nadzwyczajnym.
        \item Walne Zebranie Członków w trybie zwyczajnym jest zwoływane nie rzadziej niż raz na rok.
        \item Walne Zebranie Członków w trybie zwyczajnym jest zwoływane przez Zarząd i odbywa się w siedzibie Stowarzyszania.
    \item Dla zwołania Walnego Zebrania Członków w trybie zwyczajnym Zarząd podaje dwa terminy. Jeśli w pierwszym terminie nie zbierze się kworum, Walne Zebranie Członków odbywa się w drugim terminie.
      \begin{enumerate}
        \item Termin obrad Zarząd podaje do wiadomości wszystkich członków pocztą elektroniczną co najmniej 14 dni przed pierwszym terminem zebrania.
        \item Aby Walne Zebranie Członków mogło się rozpocząć, w pierwszym terminie wymagane jest kworum, czyli obecność co najmniej połowy ogólnej liczby zwyczajnych członków Stowarzyszenia.
        \item W drugim terminie kworum nie jest wymagane.
        \item Oba terminy Walnego Zebrania Członków w trybie zwyczajnym muszą być odległe od siebie przynajmniej 7, ale nie bardziej niż 14 dni kalendarzowych.
      \end{enumerate}
    \item Walne Zebranie Członków w trybie nadzwyczajnym może się odbyć w każdym czasie.
      \begin{enumerate}
        \item Na Walnym Zebraniu Członków zwołanym w trybie nadzwyczajnym, obraduje się tylko nad sprawami zgłoszonymi przez członków minimum na 3 dni przed zebraniem.
      \end{enumerate}
    \item Walne Zebranie Członków w trybie nadzwyczajnym jest zwoływane:
      \begin{enumerate}
        \item przez Zarząd,
        \item na wniosek Komisji Rewizyjnej,
        \item na pisemny wniosek co najmniej 1/3 ogólnej liczby członków zwyczajnych Stowarzyszenia, złożony na ręce Zarządu,
        \item Zarząd zobowiązany jest w ciągu 7 dni od otrzymania wniosku podjąć uchwałę o zwołaniu Walnego Zebrania Członków w trybie nadzwyczajnym,
        \item w przypadku braku zarządu, przez jedną trzecią Członków Zwyczajnych przez poinformowanie wszystkich członków stowarzyszenia zgodnie ze statutem.
        \end{enumerate}
    \item Dla zwołania Walnego Zebrania Członków w trybie nadzwyczajnym wnioskujący podaje dwa terminy. Jeśli w pierwszym terminie nie zbierze się kworum Walne Zebranie Członków odbywa się w drugim terminie.
      \begin{enumerate}
        \item Termin obrad podaje Zarząd do wiadomości wszystkich członków pocztą elektroniczną co najmniej 7 dni przed pierwszym terminem zebrania.
        \item Aby Walne Zebranie Członków w trybie nadzwyczajnym mogło się rozpocząć, w pierwszym terminie wymagane jest kworum, czyli obecność co najmniej połowy ogólnej liczby zwyczajnych członków Stowarzyszenia.
        \item W drugim terminie kworum nie jest wymagane.
        \item Oba terminy Walnego Zebrania Członków w trybie nadzwyczajnym muszą być odległe od siebie przynajmniej 3, ale nie bardziej niż 7 dni kalendarzowych.
      \end{enumerate}
    \item O terminie i~miejscu Walnego Zebrania Członków Zarząd informuje pocztą elektroniczną wszystkich członków Stowarzyszenia, dołączając do zawiadomienia sprawozdanie z~działalności Stowarzyszenia oraz projekt regulaminu obrad.
    \item Zarząd jest powołany do kierowania całą działalnością Stowarzyszenia zgodnie z uchwałami Walnego Zebrania Członków.
    \item W skład Zarządu wchodzi 3 równorzędnych członków.
    \item Do kompetencji Zarządu należą:
      \begin{enumerate}
        \item realizacja celów Stowarzyszenia,
        \item wykonywanie uchwał Walnego Zebrania Członków,
        \item sporządzanie planów pracy i budżetu,
        \item sprawowanie zarządu nad majątkiem Stowarzyszenia,
        \item podejmowanie uchwał o zarządzaniu majątkiem Stowarzyszenia,
        \item reprezentowanie Stowarzyszenia na zewnątrz,
        \item zwoływanie Walnego Zebrania Członków,
        \item przyjmowanie i skreślanie członków zwyczajnych,
        \item składanie sprawozdań ze swojej działalności na Walnym Zebraniu Członków,
        \item sporządzanie rocznego sprawozdania finansowego,
        \item zwalnianie w uzasadnionych przypadkach Członków Zwyczajnych z obowiązku płacenia składek w całości bądź w części.
      \end{enumerate}
    \item Posiedzenia Zarządu odbywają się w miarę potrzeb, nie rzadziej jednak niż raz na rok.
    \item Posiedzenia Zarządu zwołuje:
      \begin{enumerate}
        \item dwóch członków zarządu działających łącznie,
        \item Komisja Rewizyjna.
      \end{enumerate}
    \item Komisja Rewizyjna powoływana jest do sprawowania kontroli nad działalnością Stowarzyszenia.
    \item Komisja Rewizyjna składa się z 3 równorzędnych członków.
    \item Do kompetencji Komisji Rewizyjnej należy:
      \begin{enumerate}
        \item kontrolowanie działalności Zarządu,
        \item składanie sprawozdań z kontroli na Walnym Zebraniu Członków,
        \item prawo wystąpienia z wnioskiem o zwołanie Walnego Zebrania Członków oraz zebrania Zarządu,
        \item składanie wniosków o absolutorium dla Zarządu Stowarzyszenia,
        \item składanie sprawozdań ze swojej działalności na Walnym Zebraniu Członków,
        \item zatwierdzanie rocznego sprawozdania finansowego.
      \end{enumerate}
    \item W razie gdy skład władz Stowarzyszenia ulegnie zmniejszeniu w czasie trwania kadencji, uzupełnienie ich składu może nastąpić w drodze kooptacji, której dokonują pozostali członkowie organu, który uległ zmniejszeniu. W tym trybie można powołać nie więcej niż połowę składu organu. Kadencja dokooptowanego członka organu kończy się wraz z kadencją całego organu.
      \begin{enumerate}
        \item W przypadku, gdy skład władz Stowarzyszenia ulegnie zmniejszeniu o więcej niż połowę, Zarząd zwołuje Walne Zebranie Członków w trybie nadzwyczajnym celem uzupełnienia władz.
      \end{enumerate}
    \item {[uchylono]}
    \item Członek Zarządu bądź Komisji Rewizyjnej może zrezygnować z pełnienia funkcji poprzez złożenie oświadczenia na Walnym Zebraniu Członków.
  \end{enumerate}
\section{MAJĄTEK I FUNDUSZE}
  \begin{enumerate}
    \item Majątek Stowarzyszenia składa się z:
      \begin{enumerate}
        \item składek członkowskich,
        \item subwencji, darowizn, spadków i zapisów,
        \item wpływów z odpłatnej działalności statutowej,
        \item wpływów z ofiarności publicznej,
        \item dochodów z majątku, odsetek oraz kapitału,
        \item dotacji,
        \item wpływów z loterii, aukcji i sponsoringu,
        \item zbiórek publicznych.
      \end{enumerate}
    \item Stowarzyszenie prowadzi gospodarkę finansową zgodnie z obowiązującymi przepisami.
    \item Organem kompetentnym w zakresie zarządzania majątkiem Stowarzyszenia jest Zarząd.
    \item Zarząd zobowiązany jest dołożyć wszelkich starań w celu utrzymania zapasu środków na kontach Stowarzyszenia wystarczającego na pokrycie stałych zobowiązań Stowarzyszenia przez okres co najmniej dwóch miesięcy.
    \item Do zawierania umów, zwłaszcza w sprawach majątkowych i niemajątkowych, udzielania pełnomocnictwa i składania innych oświadczeń woli, w szczególności w sprawach majątkowych upoważnieni są:
      \begin{enumerate}
        \item dwóch członków Zarządu działających łącznie,
        \item każdy członek zarządu działający samodzielnie, jeżeli wysokość podejmowanego zobowiązania nie przekracza 512 PLN.
      \end{enumerate}
    \item Stowarzyszenie nie prowadzi działalności gospodarczej.
    \item Składki członkowskie ustanawiane są uchwałą Walnego Zebrania Członków.
  \end{enumerate}
\section{POSTANOWIENIA KOŃCOWE}
  \begin{enumerate}
    \item Uchwałę w sprawie zmiany Statutu oraz uchwałę o rozwiązaniu Stowarzyszenia podejmuje Walne Zebranie Członków kwalifikowaną większością głosów (2/3), przy obecności co najmniej połowy łącznej liczby członków Stowarzyszenia uprawnionych do głosowania, stanowiących kworum.
    \item Podejmując uchwałę o rozwiązaniu Stowarzyszenia Walne Zebranie Członków określa sposób jego likwidacji oraz przeznaczenie majątku Stowarzyszenia.
    \item W sprawach nie uregulowanych w niniejszym statucie zastosowanie mają przepisy ustawy Prawo o Stowarzyszeniach.
  \end{enumerate}
\end{document}
