\documentclass{article}
\usepackage[OT4]{fontenc}
\usepackage{polski}
\usepackage[utf8]{inputenc}

\usepackage[margin=2cm]{geometry}

\renewcommand{\thesection}{\Roman{section}.}
\renewcommand{\thesubsection}{\Roman{section}\alph{subsection}.}
\renewcommand{\labelenumi}{\arabic{enumi}.}
\renewcommand{\labelenumii}{\arabic{enumi}.\arabic{enumii}.}
\renewcommand{\labelenumiii}{\arabic{enumi}.\arabic{enumii}.\arabic{enumiii}}

\title{\huge REGULAMIN WARSZTATU}
\date{Wersja z dnia \today.}

\begin{document}
\maketitle
\section{POSTANOWIENIA OGÓLNE}
    \begin{enumerate}
  	\item Regulamin określa zasady funkcjonowania warsztatu prowadzonego przez Stowarzyszenie Hackerspace Wrocław i Centrum Przedsiębiorczości i Biznesu „DĄBIE”, przy ulicy Wróblewskiego 38, 51-627 Wrocław. 
  	\item Ilekroć w regulaminie jest mowa o:
        \begin{enumerate}
        \item \textbf{Regulaminie} - należy przez to rozumieć ten dokument,
        \item \textbf{Stowarzyszeniu} - należy przez to rozumieć Stowarzyszenie Hackerspace Wrocław, a w kwestiach wymagających podjęcia decyzji jego Zarząd,
        \item \textbf{Centrum} - należy przez to rozumieć Centrum Przedsiębiorczości i Biznesu „DĄBIE”,
        \item \textbf{Operatorze warsztatu} - należy przez to rozumieć członków Stowarzyszenia i pracowników Centrum wyznaczonych do nadzoru nad pracą w warsztacie, a pod ich nieobecność - innych członków Stowarzyszenia,
        \item \textbf{Warsztacie} - należy przez to rozumieć pomieszczenia warsztatowe,  
        \item \textbf{Użytkowniku warsztatu} - należy przez to rozumieć osobę obecną w jednym z pomieszczeń warsztatu, która nie jest członkiem Stowarzyszenia ani pracownikiem Centrum,
        \item \textbf{Stanowisku warsztatowym} - należy przez to rozumieć wydzieloną część warsztatu dedykowaną specyficznemu rodzajowi prac (np. elektroniczne, mechaniczne), wyposażoną w narzędzia i instrukcję stanowiskową,
        \item \textbf{Instrukcji stanowiskowej} - należy przez to rozumieć odrębny dokument umieszczony na stanowisku warsztatowym, precyzujący zasady obowiązujące na tym stanowisku.
        \end{enumerate}
    \item Przed skorzystaniem z Warsztatu użytkownik jest zobligowanany do zapoznania się z regulaminem. Korzystając z Warsztatu użytkownik zobowiązuje się do przestrzegania Regulaminu oraz stosowania się do poleceń Operatora warsztatu.
    \item Nieznajomość Regulaminu nie zwalnia od odpowiedzialności za szkody wynikające z niestosowania się do jego zapisów.
    	\item \textbf{Warsztat jest monitorowany}. Administratorem nagrań jest Stowarzyszenie.
    \end{enumerate}
\section{POSTANOWIENIA SZCZEGÓŁOWE}
    \begin{enumerate}
    \item Dostęp do warsztatu jest udzielany odpłatnie,  po wcześniejszym szkoleniu wstępnym prowadzonym przez Operatora warsztatu, na podstawie wykupionego za pośrednictwem Centrum abonamentu. 
    \item W wypadku gdy dostęp na podstawie abonamentu jest udzielany grupie, grupa jest zobowiązana do wyznaczenia osoby odpowiedzialnej. Osoba ta stanowi punkt kontaktowy pomiędzy grupą, a Operatorem warsztatu oraz jest zobowiązana do nadzoru nad przestrzeganiem regulaminu w trakcie prac.
    \item Członkowie Stowarzyszenia oraz ich goście są zwolnieni z opłat za dostęp do warsztatu. Członkowie Stowarzyszenia ponoszą odpowiedzialność za swoich gości.
    \item Warsztat jest dostępny dla użytkowników w godzinach pracy Centrum. Dostęp poza tymi godzinami jest możliwy po wcześniejszym ustaleniu z Operatorem warsztatu.
    \item Naruszenie postanowień regulaminu może skutkować odebraniem dostępu, bez możliwości zwrotu opłaty abonamentowej.
    \item Dostęp do warsztatu może być czasowo wstrzymywany ze względu na organizowane przez Stowarzyszenie i Centrum wydarzenia oraz warsztaty. Stowarzyszenie informuje o wszystkich ograniczeniach za pośrednictwem udostępnionego elektronicznie kalendarza.
    \end{enumerate}
\subsection{REGUŁY PORZĄDKOWE}
    \begin{enumerate}
    \item Użytkownicy warsztatu są zobowiązani do utrzymania porządku na używanych stanowiskach warsztatowych. Po zakończeniu pracy na stanowisku należy je każdorazowo uprzątnąć. W innym wypadku użytkownicy mogą zostać obciążeni kosztami sprzątania stanowiska.
    \item W wypadku zastania nieuporządkowanego przez poprzedniego użytkownika stanowiska należy zgłosić ten incydent Operatorowi warsztatu. Zabronione jest rozpoczynanie pracy na nieuprzątniętym stanowisku.
    \end{enumerate}
\subsection{BEZPIECZEŃSTWO I HIGIENA PRACY}
    \begin{enumerate}
    \item Użytkownicy warsztatu są zobowiązani wykonywać pracę w sposób zgodny z przepisami przeciwpożarowymi, BHP i zasadami zdrowego rozsądku oraz zobowiązani stosować się do wydawanych w tym zakresie poleceń i wskazówek Operatora warsztatu.
    \item Użytkownicy warsztatu są zobowiązani do korzystania ze środków ochrony osobistej. \textbf{Korzystanie ze stanowisk bez wykorzystania sprawnych środków ochrony osobistej jest surowo zabronione!}
    \item Przed skorzystaniem ze stanowiska, użytkownik jest zobowiązany do zapoznania się z dostępną instrukcją stanowiskową i podporządkowania się jej zaleceniom. Użytkownicy nie posiadający doświadczenia w użyciu konkretnych narzędzi mają obowiązek zgłosić się do Operatora w celu wcześniejszego przeszkolenia.
    \item Użytkownicy są zobowiązani do dbania o należyty stan narzędzi i maszyn, używania ich zgodnie z przeznaczeniem, bez narażania ich na przyspieszone zużycie lub zniszczenie. Przed każdym użyciem użytkownik jest zobowiązany do sprawdzenia stanu technicznego narzędzi na stanowisku.
    \item W wypadku zastania narzędzi w stanie nie nadającym się do ich użycia (np. narzędzia stępione, z uszkodzoną izolacją) należy niezwłocznie poinformować o tym fakcie Operatora warsztatu oraz wyłączyć narzędzie z użycia odpowiednio je oznaczając. \textbf{Użycie uszkodzonych narzędzi jest surowo zabronione!}
    \end{enumerate}		
\section{POSTANOWIENIA KOŃCOWE}
    \begin{enumerate}
    \item Mając na uwadze charakter w jaki prowadzona jest działalność warsztatu (udostępnianie przestrzeni warsztatowej wyposażonej w narzędzia, bez stałego nadzory) oraz nieodłączne ryzyko związane z korzystaniem z narzędzi i maszyn, nie jest możliwe pełne wyeliminowanie ryzyka wypadku. W związku z tym Stowarzyszenie oraz Centrum nie ponoszą odpowiedzialności za ewentualne szkody ponoszone przez użytkowników wynikające z korzystania z warsztatu.
    \item Użytkownicy mogą zostać pociągnięci do odpowiedzialności (w tym finansowej) za spowodowane szkody wynikające z korzystania z warsztatu.
    \item Kwestie nieuregulowane w regulaminie rozstrzyga Zarząd Stowarzyszenia.
    \end{enumerate}
\end{document}
